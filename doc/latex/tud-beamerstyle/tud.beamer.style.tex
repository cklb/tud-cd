% -*- LaTeX -*-
\usepackage{xcolor}
\usepackage{listings}

\title[Präsentationsvorlagen im CD]{Präsentationsvorlagen\protect\\\mdseries im CD der TU Dresden\strut}
\subtitle{Beamer-Stil}
\author{Tobias Schlemmer}
\einrichtung{Einrichtung}
\fachrichtung{Fachrichtung}
\institut{Institut für Algebra}
\professur{Professur}

\newcommand*\inmm[1]{\pgfmathsetmacro\inmmwert{#1 / 1mm}\inmmwert}
\makeatletter
\newcommand*\inpt[1]{\setlength\@tempdima{#1}\the\@tempdima}
\makeatother

\AtBeginSection[]{\partpage{\usebeamertemplate***{part page}}} 
\begin{document}
\section*{Vorwort}
\subsection*{Hinweis auf die Dokumentation}
% \partpage{\usebeamertemplate***{part page}}
\begin{frame}{Dokumentation}
  Die Dokumentation befindet sich in der Datei
  \url{beamer-org-mode-demo.pdf}.

  \vfill
  Es gibt Seitenzahlen und Foliennummern, die sich in diesen
  Beispiel in unterschiedlichen Stilen abwechseln. Diese Seite zeigt die Vorgabe: Folie ohne Anhang.

  \vfill
  {}\hfill los gehts\dots
\end{frame}
\setbeamertemplate{tud background}[image/shaded]{Seminarraum.jpg}{0.7}
\maketitle
\setbeamertemplate{page number in footline}[page][text and total]
\frame{\frametitle{Inhalt}\tableofcontents}
\setbeamertemplate{tud background}[image/shaded]{Posterflur.jpg}{0.6}
\section{Geschichte}
\setbeamertemplate{page number in footline}[page][text and filetotal]
\subsection{(eher ein Roman keine Präsentation)}
\begin{frame}\frametitle{2005 – Einführung des Corporate Designs}\framesubtitle{Kein \LaTeX{}}
  Bei der Einführung des Corporate Designs der TU Dresden 2005 hielt
  man es zunächst nicht für erforderlich, „Nischenprodukte“ wie
  \LaTeX{} zu unterstützen. Dabei wurde übersehen, dass in einigen
  Fachbereichen z.\,B.\ der Mathematik, Logik und Linguistik \TeX{} und
  die darauf aufbauenden Formate die dominante Form der
  Textverarbeitung darstellen.
\end{frame}

\setbeamertemplate{page number in footline}[page][text]
\begin{frame}[allowframebreaks]\frametitle{Erste \LaTeX{}-Klassen}
  Die aufkommende Nachfrage führte dazu, dass relativ schnell
  \LaTeX-Klassen für die wichtigsten Textdokumente nachgeliefert
  wurden. Später folgte auch eine \texttt{tudbeamer}"=Klasse. Diese
  kam aber zu spät, war zu unflexibel, benutzte veraltete Pakete und
  war zu sehr auf hochsprachlichen Elementen von \LaTeX{} aufgebaut,
  so dass sie den Ansprüchen an eine moderne Kommunikation nicht
  gerecht werden konnte. Das späte Erscheinen einer zentralen Vorlage
  führte auch dazu, dass an vielen Instituten (z.\,B. ZIH,
  Nachrichtentechnik) eigene Stile entwickelt wurden, die den Vorgaben
  meist nicht gereicht wurden. Viele dieser Stile sind bis heute im
  Einsatz.
\end{frame}

\setbeamertemplate{page number in footline}[page][total]
\begin{frame}[allowframebreaks]\frametitle{2010 – Evaluierung der Fachrichtung Mathematik}
  Als 2010 die Fachrichtung Mathematik evaluiert wurde, sollten sich
  die einzelnen Mitarbeiter und Arbeitsgruppen in einer Poster-Session
  vorstellen. Laut Vorgabe musste das Corporate Design sehr streng
  umgesetzt werden. Dies führte zu einem erbitterten Streit zwischen
  der WYSIWYG- und der \LaTeX"=Fraktion. Die einen warfen den anderen
  ein zu kompliziertes Werkzeug vor, die anderen glaubten nicht so
  recht an die Zuverlässigkeit und optische Qualität der anderen.

  Innerhalb von kürzester Zeit wurde Anhand des CD-Handbuches eine
  \LaTeX"=Vorlage erstellt, die – angereichert um Ti\emph{k}Z und
  andere Pakete die Erstellung von Postern demonstrierte. Dabei trat
  das erste Problem auf: Die PowerPoint-Vorlagen waren etwas größer
  und entsprachen nicht den Proportionen der DIN A"-Reihe, als die
  \LaTeX"=Vorlagen. Das fiel im WYSIWYG"=Lager zunächst nicht auf,
  denn die InDesign"=Vorlagen entsprachen den Vorgaben und die
  PowerPoint"=Poster erhielten alle in InDesign ihren Feinschliff.

  Die Ursache wurde erst offenbar, nachdem zu viele Poster schon
  fertig waren, als dass man sie hätte nachträglich noch ändern
  können. Die \LaTeX{}-Vorlage wurde erweitert, um beide Papierformate
  zu unterstützen. Da die \LaTeX{}-Poster von mehr Platz ausgingen,
  als ihnen zustand, ließen sie sich nicht ohne größeren Aufwand in
  das geforderte Format bringen. Es wurde stattdessen ein Kompromiss
  gefunden. Mit einfachen \LaTeX{}-Boardmitteln wurden die
  Poster-Inhalte ausgeschnitten und verkleinert in neue Poster
  eingefügt. Die relativ wenigen InDesign"=Poster wurden in der
  Schriftgröße angepasst und der zusätzlich gewonnene Platz sinnvoll
  verteilt.%
  \setbeamertemplate{page number in footline}[page][only]

  Durch die sehr konstruktive Zusammenarbeit der zuständigen Akteure
  entwickelte sich ein Vertrauensverhältnis und eine Akzeptanz der
  jeweils anderen Arbeitsweise. Dazu trug auch bei, dass es auf allen
  Seiten Beiträge gab, die sich von Kurzartikeln in den ersten
  Entwürfen zu ansprechenden Postern entwickelten.
\end{frame}
\begin{frame}\frametitle{Kurze Unterbrechung}
  \framesubtitle{Seitenzahlen}
  Seitenzahlen \uncover<2->{zählen} \uncover<3->{jede} \uncover<4->{PDF"=Seite}, Foliennummern nicht.
\end{frame}
\setbeamertemplate{page number in footline}[frame][text and total]
\begin{frame}[allowframebreaks]
  \frametitle{tudbeamer} Die Übertragung von Erkenntnissen aus der
  Postergestaltung auf Präsentationen förderte einige Probleme mit der
  Klasse \texttt{tudbeamer} zu Tage, deren Behebung zur Entwicklung
  des TUD"=Beamerstils geführt hat.
  \begin{block}{Nachteile von tudbeamer.cls}
    \begin{itemize}
    \item Keine Benutzung von beamerarticle.sty möglich
    \item Lädt veraltetes Paket ngerman.sty und provoziert Inkompatibilitäten
    \item Benutzt Tabellen für das Layout und beißt sich mit xcolor.sty und colortbl.sty
    \item Fehlerhafter Zeilenabstand zwischen vorletzter und letzter Titelzeile
    \item Monolithisch: Arbeit an Ford und Layout doppelt sich – schwer zu warten.
    \item Falscher unterer und rechter Rand
    \end{itemize}
  \end{block}
\end{frame}
\setbeamertemplate{page number in footline}[frame][text and filetotal]
\begin{frame}[allowframebreaks]
  \frametitle{2012 – Modularisierung und freie Interpretation}

  Der Beamerstil war zunächst nur eine Aufteilung und Berichtigung des
  Quelltextes der ursprünglichen Klasse \texttt{tudbeamer}, die einige
  Fehler der Klasse zu umgehen suchte und die Bedienung näher an die
  der Beamerklasse rückte. Dadurch sollte auch die Notwendigkeit für
  Support reduziert werden, da Beamer eine sehr gut dokumentierte und
  oft benutze und diskutierte Klasse im Internet ist. Kurz – es wurden
  hauptsächlich die Probleme behoben, die im Widerspruch zur
  offiziellen Beamer"=Dokumentation standen.

  2012 wurde der Beamer"=Stil in viele kleine einzeln konfigurierbare
  Elemente aufgespalten. Dies ermöglichte es einerseits, die spärlich
  dokumentierte offizielle Präsentationsvorlage freier auszulegen,
  andererseits ermöglichte sie es, weitere Optionen zur Verfügung zu
  stellen, die Beamer bereitstellt, aber unter PowerPoint so nicht zur
  Verfügung stehen.

  Hinzu kamen unter anderem eine mögliche optische Trennung zwischen
  Fußzeile und Inhalt, sowie die Möglichkeit, die Fußzeile auf ein
  notwendiges Minimum zu beschränken. Insbesondere der Große
  Platzbedarf der Fußzeile wurde oft kritisiert.
\end{frame}
\setbeamertemplate{page number in footline}[frame][text and total and overlay]
\begin{frame}
  \frametitle{2018 – Neues Corporate Design} Mit der Einführung des
  überarbeiteten Corporate Designs wurde 2018 auch der Beamer-Stil
  überarbeitet. Um die Nutzer nicht gänzlich von ihrer Geschichte
  abzuschneiden, bleiben die Einstellungen der Vorgängerversionen
  verfügbar. Mit der Unterstützung von \texttt{\textbackslash
    institute} und weiteren Teilvorlagen rückt der Stil dabei noch
  näher an die Ideen der originalen Beamer"=Klasse.
\end{frame}
\begin{frame}\frametitle{Noch eine Info"=Pause}
  \framesubtitle{Overlay-Nummern}

  \uncover<2->{\alert<2>{Zum Aufbauen}} %
  \uncover<3->{\alert<3>{von Folien}} %
  \uncover<4->{\alert<4>{kann man}} %
  \uncover<5->{\alert<5>{sich auch}} %
  die so genannten \alert<6>{Overlay}"=Nummern anzeigen lassen. Der
  Name kommt vom Auflegen zusätzlicher Folien beim Polylux
  (Tageslichtprojektor).
\end{frame}

\setbeamertemplate{page number in footline}[frame][text and filetotal and overlay]
\setbeamertemplate{tud background}[image/shaded]{Schreibwerkzeug.jpg}{0.6}
\section{Benutzung}

\begin{frame}[allowframebreaks]{Einbindung}
  Die Klasse kann einfach eingebunden werden:
  \begin{block}{}
    \texttt{\textbackslash usetheme\{tud\}} oder
    \texttt{\textbackslash usetheme[cd2018]\{tud\}}    
  \end{block}
  Und schon erscheint die Präsentation im Corporate Design der TU Dresden.

  \begin{block}{}
    \texttt{\textbackslash usetheme[cd2012ts]\{tud\}}
  \end{block}
  Dabei werden die Schriftarten in folgender Reihenfolge gesucht:
  \begin{enumerate}
  \item Schriften des \texttt{tudscr}-Paketes
  \item<+-> \texttt{tudscrold}-Schriften
  \item<+-> Schriften der alten TUD-\LaTeX-Klassen
  \end{enumerate}
  \framebreak
  Es gibt weitere Paketoptionen, mit denen das Layout grob angepasst
  werden kann.  Die meisten wurden von \texttt{tudbeamer.cls}
  geerbt. Die beiden Optionen „nogerman“ und „german“
  entfallen. Verwenden Sie stattdessen bitte
  \begin{block}{}
    \textbackslash usepackage[ngerman]\{babel\}
  \end{block}
  für Ihren Deutschen Text. Das Paket \texttt{(n)german.sty} ist
  veraltet und zu einigen Paketen inkompatibel.
\end{frame}

\setbeamertemplate{page number in footline}[frame][text]
\begin{frame}[allowframebreaks]{Optionen}
  \begin{description}
    \item[heavyfont] Stärkere Schriften
    \item[nodin] Lade keine {\dinfamily DIN bold}
    \item[beamerfont] Keine TU"=Schriften
    \item[serifmath] Benutze die vorgegebene Serifenschrift für mathematische Formeln
    \item[noheader] Keine Kopfzeile mit Logo (außer Titelseite)
    \item[smallrightmargin] Benutze verringerten rechten Rand von tudbeamer.cls
    \item[pagenum] Seitennummern in der Fußzeile
    \item[nosectionnum] Keine Abschnittsnummern in Folienüberschriften
    \item[navbar] Navigationszeile
    \item[ddc] Logo von DRESDEN"=concept als Zweitlogo auf der
      Titelseite (benötigt Logo"=Datei von tudbeamer.cls). Diese
      Option ist für Präsentationen im Zusammenhang mit DRESDEN"=concept
      vorbehalten.
    \item[ddcfooter] Logo von DRESDEN"=concept in der Fußzeile der
      Titelseite (Voreinstellung, benötigt Logo"=Datei von
      tudbeamer.cls). Diese Option ist für alle Präsentationen der TUD
      gedacht, die nicht im Rahmen von DRESDEN"=concept abgehalten
      werden.
    \item[noddc] Es wird kein Logo von DRESDEN"=concept angezeigt
    \item[cd2012ts] Die letzte Fassung des Beamer-Stils vor 2018 wird verwendet
    \item[cd2018] Das CD von 2018 wird umgesetzt, mit einigen \hyperref[baustellen]{Baustellen}
  \end{description}
\end{frame}

\begin{frame}[allowframebreaks]\frametitle{Hinweise}
  \begin{itemize}
  \item Die Titelseite erzeugen Sie mit \texttt{\textbackslash maketitle}.
  \item Die Abschnittsüberschriftsseite wird mit \texttt{\textbackslash partpage\{inhalt\}} erzeugt
  \item Alle Einstellmöglichkeiten werden in den Dateien \texttt{beamer*.sty}
    definiert.\footnote{Das ist die Ausrede derer, die zu faul sind,
      eine ordentliche Dokumentation zu schreiben, oder die aus anderen Gründen keine Zeit haben.}
  \item Alle Fragen, die dann noch bleiben, können gerne auf \url{http://github.com/tud-cd/tud-cd} als neues „Issue“ eröffnet und diskutiert werden.
  \item Das Dokument \texttt{beamer-org-mode-demo.pdf} enthält eine Kurzreferenz.
  \item Darüber hinaus wäre sicherlich eine ausführliche Dokumentation der einzelnen Einstellungen sinnvoll. Wer dort helfen will, kann gern auch gern ein „Issue“ auf \url{http://github.com/tud-cd/tud-cd} eröffnen.
  \item Erweiterungen, Anpassungen und Fehlerkorrekturen werden auf
    \url{http://github.com/tud-cd/tud-cd} gern entgengenommen.
  \end{itemize}
\end{frame}

\setbeamertemplate{page number in footline}[frame][text and overlay]
\setbeamertemplate{tud background}[image/shaded]{Baustelle.jpg}{0.6}
\section{Über den Beamer-Stil}
\begin{frame}{Konzeptionelles}
  \begin{itemize}
  \item Optionale Optische Kompatibilität zu vorheriger Fassung
  \item<+-> Algorithmisches Layout
  \item<+-> Volle Beamer-Unterstützung angestrebt (z.\,B. weitere Seitenverhältnisse, theorem-Blöcke u.\,ä.)
  \item<+> Metrische Seitengröße (im Gegensatz zu 10/7,5 Zoll bei PowerPoint)
  \item<+> Seitengröße entsprechend Schriftgröße (bei PowerPoint anders herum)
  \item<+-> PowerPoint-Vorlagen approximativ bei 8\,pt Brotschrift
  \end{itemize}
\end{frame}
\setbeamertemplate{page number in footline}[frame][text and total]
\begin{frame}{Baustellen}\label{baustellen}
  \begin{itemize}
  \item Pixelgenaue PowerPoint-Vorlage-Rekonstruktion für gemischte Präsentationen PowerPoint/\LaTeX.
    \begin{itemize}
    \item Neue Stiloption \texttt{ppt},
    \item Hack, um an den Parameter \texttt{aspectratio} zu kommen,
    \item Hack zum Aushebeln der Beamer-Seitengröße
    \item Templates \texttt{page layout/cd2018 ppt43}, \texttt{page
        layout/cd2018 ppt169} und \texttt{page layout/cd2018 ppt1610}
      hinzufügen und einbinden.
    \item Anpassungen an Templates für Titelseite, Abschnittsseite, um
      weitere Maße konfigurierbar zu machen.
    \item Entwicklung weiterer Paketoptionen für die Schriftvorlagen
    \end{itemize}
  \end{itemize}
  \begin{enumerate}     
  \item \dots\setcounter{enumi}{8}
  \item Bei langen Aufzählungen 
  \item gibt es
  \item Kollisionen.
  \end{enumerate}
\end{frame}

\setbeamertemplate{page number in footline}[frame][text and filetotal]
\setbeamertemplate{tud background}[image/shaded]{Ausfahrt.jpg}{0.6}
\section{Abschlussfolie}
\begin{frame}
  \begin{block}{}
    \centerline{\huge\textbf{Viel Spaß!}}
  \end{block}
  \vfill P.S.: Die Beigelegte Präsentation ist ein Beispiel für die
  Verwendung der Klasse, aber als Präsentation völlig
  ungeeignet. Tipps für Ihre Präsentation können sie u.\,a.\ der Datei
  beameruserguide.pdf ihrer \TeX"=Installation entnehmen.
\end{frame}
\setbeamertemplate{page number in footline}[frame][text and total and overlay]
\frame{\frametitle{Seitennummerndemonstrationsseite I}text \only<2->{and total} \only<3->{and overlay}}
\setbeamertemplate{page number in footline}[frame][text and filetotal and overlay]
\frame{\frametitle{Seitennummerndemonstrationsseite II}text \only<2->{and filetotal} \only<3->{and overlay}}
\setbeamertemplate{page number in footline}[frame][text]
\frame{\frametitle{Seitennummerndemonstrationsseite III}text}
\setbeamertemplate{page number in footline}[frame][text and overlay]
\frame{\frametitle{Seitennummerndemonstrationsseite IV}text \only<2->{and} \only<3->{overlay}}
\setbeamertemplate{page number in footline}[frame][total]
\frame{\frametitle{Seitennummerndemonstrationsseite V}total}
\setbeamertemplate{page number in footline}[frame][filetotal]
\frame{\frametitle{Seitennummerndemonstrationsseite VI}filetotal bedeutet: inklusive Anhang}
\setbeamertemplate{page number in footline}[frame][total and overlay]
\frame{\frametitle{Seitennummerndemonstrationsseite VII}total \only<2->{and} \only<3->{overlay}}
\setbeamertemplate{page number in footline}[frame][filetotal and overlay]
\frame{\frametitle{Seitennummerndemonstrationsseite VIII}filetotal \only<2->{and} \only<3->{overlay}}
\setbeamertemplate{page number in footline}[frame][only]
\frame{\frametitle{Seitennummerndemonstrationsseite IX}only}
\setbeamertemplate{page number in footline}[frame][overlay]
\frame{\frametitle{Seitennummerndemonstrationsseite X}o\only<+->v\only<+->e\only<+->r\only<+->l\only<+->a\only<+->y

  Es folgt der Anhang}

\setbeamertemplate{page number in footline}[frame][text and total]
\setbeamertemplate{tud background}[shaded]
\appendix
\section{Hilfsseiten für die Erstellung der Vorlagen}

\begin{frame}
  \frametitle{Maßsystem}
  \framesubtitle{zum Testen}

  Eine Seite ist $\inmm\paperwidth\times\inmm{\paperheight}$\,mm${}^2$
  groß.  $1\,$pt ist $\inmm{1pt}$\,mm und $2660$\,pt sind
  $\inmm{2660pt}$\,mm groß.  $1\,$pc ist $\inmm{1pc}$\,mm und $266$\,pc
  sind $\inmm{266pc}$\,mm groß.  $1\,$in ist $\inmm{1in}$\,mm und
  $26.6$\,in sind $\inmm{26.6in}$\,mm groß.  $1\,$in ist
  $\inpt{1in}$\,pt und $26.6$\,in sind $\inpt{26.6in}$\,pt groß.  1
  Pariser Punkt beträgt $\inpt{0.376mm}$ seit 1975 $\inpt{0.375mm}$
  (Änderung hat sich nicht durchgesetzt).

  1\,bp (DTP-Punkt) ist $\inpt{1bp}$.
  
  \ss eine Text-Zahl 1234567890.

  \[
    \int_{[0,1]\setminus \{0\}} e^{2\pi i} = 1 \mathbf{a}\mathsf{a}{a}
  \]
  \edef\tempa{.36226328 * \the\tudbeamerlogoheight}
\end{frame}
\begin{frame}{Stichpunkte usw.}
  Dies ist ein Blindtext, er wird erweitert, bis die zweite
  Zeile anfängt, so dass man den Zeilenumbruch sieht
  \begin{itemize}
  \item
    Dies ist ein Blindtext, er wird erweitert, bis die zweite
    Zeile anfängt, so dass man den Zeilenumbruch sieht
    \begin{itemize}
    \item Dies ist ein Blindtext, er wird erweitert, bis die zweite
      Zeile anfängt, so dass man den Zeilenumbruch sieht
      \begin{itemize}
      \item Dies ist ein Blindtext, er wird erweitert, bis die zweite
        Zeile anfängt, so dass man den Zeilenumbruch sieht
      \end{itemize}
    \end{itemize}
  \end{itemize}
  Dies ist ein Blindtext, er wird erweitert, bis die zweite
  Zeile anfängt, so dass man den Zeilenumbruch sieht
  \begin{enumerate}
  \item   Dies ist ein Blindtext, er wird erweitert, bis die zweite
    Zeile anfängt, so dass man den Zeilenumbruch sieht
    \begin{enumerate}
    \item Dies ist ein Blindtext, er wird erweitert, bis die zweite
      Zeile anfängt, so dass man den Zeilenumbruch sieht
      \begin{enumerate}
      \item Dies ist ein Blindtext, er wird erweitert, bis die zweite
        Zeile anfängt, so dass man den Zeilenumbruch sieht
      \end{enumerate}
    \end{enumerate}
  \end{enumerate}
  Ende
\end{frame}

\end{document}

%%% Local Variables: 
%%% mode: latex
%%% TeX-master: t
%%% End: 
