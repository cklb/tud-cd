\documentclass[a5paper,fontsize=10pt,twoside,DIV12,pagesize]{scrartcl}
%\documentclass[a4paper,twoside,12pt]{amsart}
	
% Dieses Dokument demonstriert die Wirkungsweise des Preprint-Paketes f�r das
% Institut f�r Algebra an einem minimalistischen Beispiel. Eine genauere 
% Dokumentation der m�glichen Befehle findet man in dem anderen Beispiel.
%
% Fehler, Probleme, W�nsche, Verbesserungen bitte an mich richten:
% mike.behrisch@mailbox.tu-dresden.de
%

% Das Paket wird in der Pr�ambel irgendwo unter den anderen Paketen und
% Definitionen geladen (hier nur ein Beispiel)
\usepackage[latin1]{inputenc}
\usepackage[T1]{fontenc} 
\usepackage[german,british]{babel}

\usepackage[tudfonts]{algpreprint}  % "[tudfonts]" weglassen, wenn Fonts nicht
                                    % installiert sind

%\usepackage{pdfpages}  % F�r Variante 3 (siehe unten)

\begin{document}
\selectlanguage{german} % Falls der Titel etc. in deutscher Sprache ist 

% notwendige Befehle:
%--------------------
\Autor{Der Verursacher.}
\Titel[Manchmal einen Kurztitel]{%
       Jede Ver�ffentlichung braucht einen Titel:\\ und einen Untertitel}
\MathAlNummer{??}       % Einstellen der Preprintnummer
                       
% optionale Befehle:
%--------------------
%\AlsManuskriptGedruckt  % falls dieser Text auf dem Preprint erscheinen soll;
%                        % wird der Befehl nicht verwendet, so erscheint dieser
%                        % Text nicht.
%\TechnicalReport        %  --> Ersetze "PREPRINT" durch "TECHNICAL REPORT"
%\TUDfonts               %  --> Aktiviere Benutzung der TUD-Schriftarten 
%                        %      (Univers und Din bold), m�ssen dazu installiert
%                        %      sein, Default = keine TUD-Fonts
%\TitelFussnote{Ein Fu�notentext zum Titel (optional), Standard=leer} 

% Hauptbefehl:
%--------------
\Deckblatt              % sonst erscheint kein Deckblatt

%-------------------------------------------------------
% Hier kommt der eigentliche Inhalt der Publikation.
%-------------------------------------------------------
%
% 3 M�glichkeiten werden demonstriert:

% Variante 1: Die Publikation direkt schreiben:
%~~~~~~~~~~~~
\selectlanguage{british}
\author{Someone, who calls himself a mathematician}
\title{On the social life of algebras and graphs}
\date{\today}
\maketitle

An algebra goes shopping and meets a homomorphism \ldots 


% Variante 2: Die Publikation bis auf die Pr�ambel �ber eine Datei einbinden:
%~~~~~~~~~~~~
%\input{mynewresult.tex}

% Variante 3: Die fertige Publikation einbinden:
%~~~~~~~~~~~~
%\includepdf[pages=-]{mypublication.pdf}
% Nachteil: Hyperref-Optionen, insbesondere Links werden nicht �bernommen.

\end{document}